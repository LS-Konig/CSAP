% Options for packages loaded elsewhere
% Options for packages loaded elsewhere
\PassOptionsToPackage{unicode}{hyperref}
\PassOptionsToPackage{hyphens}{url}
\PassOptionsToPackage{dvipsnames,svgnames,x11names}{xcolor}
%
\documentclass[
  12pt,
  a4paper,
  DIV=11,
  numbers=noendperiod]{scrartcl}
\usepackage{xcolor}
\usepackage[top = 2cm,bottom = 2cm,left = 2.5cm,right = 2.5cm,footskip =
20pt]{geometry}
\usepackage{amsmath,amssymb}
\setcounter{secnumdepth}{5}
\usepackage{iftex}
\ifPDFTeX
  \usepackage[T1]{fontenc}
  \usepackage[utf8]{inputenc}
  \usepackage{textcomp} % provide euro and other symbols
\else % if luatex or xetex
  \usepackage{unicode-math} % this also loads fontspec
  \defaultfontfeatures{Scale=MatchLowercase}
  \defaultfontfeatures[\rmfamily]{Ligatures=TeX,Scale=1}
\fi
\usepackage{lmodern}
\ifPDFTeX\else
  % xetex/luatex font selection
\fi
% Use upquote if available, for straight quotes in verbatim environments
\IfFileExists{upquote.sty}{\usepackage{upquote}}{}
\IfFileExists{microtype.sty}{% use microtype if available
  \usepackage[]{microtype}
  \UseMicrotypeSet[protrusion]{basicmath} % disable protrusion for tt fonts
}{}
\usepackage{setspace}
\makeatletter
\@ifundefined{KOMAClassName}{% if non-KOMA class
  \IfFileExists{parskip.sty}{%
    \usepackage{parskip}
  }{% else
    \setlength{\parindent}{0pt}
    \setlength{\parskip}{6pt plus 2pt minus 1pt}}
}{% if KOMA class
  \KOMAoptions{parskip=half}}
\makeatother
% Make \paragraph and \subparagraph free-standing
\makeatletter
\ifx\paragraph\undefined\else
  \let\oldparagraph\paragraph
  \renewcommand{\paragraph}{
    \@ifstar
      \xxxParagraphStar
      \xxxParagraphNoStar
  }
  \newcommand{\xxxParagraphStar}[1]{\oldparagraph*{#1}\mbox{}}
  \newcommand{\xxxParagraphNoStar}[1]{\oldparagraph{#1}\mbox{}}
\fi
\ifx\subparagraph\undefined\else
  \let\oldsubparagraph\subparagraph
  \renewcommand{\subparagraph}{
    \@ifstar
      \xxxSubParagraphStar
      \xxxSubParagraphNoStar
  }
  \newcommand{\xxxSubParagraphStar}[1]{\oldsubparagraph*{#1}\mbox{}}
  \newcommand{\xxxSubParagraphNoStar}[1]{\oldsubparagraph{#1}\mbox{}}
\fi
\makeatother


\usepackage{longtable,booktabs,array}
\usepackage{calc} % for calculating minipage widths
% Correct order of tables after \paragraph or \subparagraph
\usepackage{etoolbox}
\makeatletter
\patchcmd\longtable{\par}{\if@noskipsec\mbox{}\fi\par}{}{}
\makeatother
% Allow footnotes in longtable head/foot
\IfFileExists{footnotehyper.sty}{\usepackage{footnotehyper}}{\usepackage{footnote}}
\makesavenoteenv{longtable}
\usepackage{graphicx}
\makeatletter
\newsavebox\pandoc@box
\newcommand*\pandocbounded[1]{% scales image to fit in text height/width
  \sbox\pandoc@box{#1}%
  \Gscale@div\@tempa{\textheight}{\dimexpr\ht\pandoc@box+\dp\pandoc@box\relax}%
  \Gscale@div\@tempb{\linewidth}{\wd\pandoc@box}%
  \ifdim\@tempb\p@<\@tempa\p@\let\@tempa\@tempb\fi% select the smaller of both
  \ifdim\@tempa\p@<\p@\scalebox{\@tempa}{\usebox\pandoc@box}%
  \else\usebox{\pandoc@box}%
  \fi%
}
% Set default figure placement to htbp
\def\fps@figure{htbp}
\makeatother


% definitions for citeproc citations
\NewDocumentCommand\citeproctext{}{}
\NewDocumentCommand\citeproc{mm}{%
  \begingroup\def\citeproctext{#2}\cite{#1}\endgroup}
\makeatletter
 % allow citations to break across lines
 \let\@cite@ofmt\@firstofone
 % avoid brackets around text for \cite:
 \def\@biblabel#1{}
 \def\@cite#1#2{{#1\if@tempswa , #2\fi}}
\makeatother
\newlength{\cslhangindent}
\setlength{\cslhangindent}{1.5em}
\newlength{\csllabelwidth}
\setlength{\csllabelwidth}{3em}
\newenvironment{CSLReferences}[2] % #1 hanging-indent, #2 entry-spacing
 {\begin{list}{}{%
  \setlength{\itemindent}{0pt}
  \setlength{\leftmargin}{0pt}
  \setlength{\parsep}{0pt}
  % turn on hanging indent if param 1 is 1
  \ifodd #1
   \setlength{\leftmargin}{\cslhangindent}
   \setlength{\itemindent}{-1\cslhangindent}
  \fi
  % set entry spacing
  \setlength{\itemsep}{#2\baselineskip}}}
 {\end{list}}
\usepackage{calc}
\newcommand{\CSLBlock}[1]{\hfill\break\parbox[t]{\linewidth}{\strut\ignorespaces#1\strut}}
\newcommand{\CSLLeftMargin}[1]{\parbox[t]{\csllabelwidth}{\strut#1\strut}}
\newcommand{\CSLRightInline}[1]{\parbox[t]{\linewidth - \csllabelwidth}{\strut#1\strut}}
\newcommand{\CSLIndent}[1]{\hspace{\cslhangindent}#1}



\setlength{\emergencystretch}{3em} % prevent overfull lines

\providecommand{\tightlist}{%
  \setlength{\itemsep}{0pt}\setlength{\parskip}{0pt}}



 


\usepackage{setspace}
\setlength{\parindent}{15pt}
\usepackage[font=footnotesize]{caption}
\KOMAoption{captions}{tableheading}
\makeatletter
\@ifpackageloaded{caption}{}{\usepackage{caption}}
\AtBeginDocument{%
\ifdefined\contentsname
  \renewcommand*\contentsname{Table of contents}
\else
  \newcommand\contentsname{Table of contents}
\fi
\ifdefined\listfigurename
  \renewcommand*\listfigurename{List of Figures}
\else
  \newcommand\listfigurename{List of Figures}
\fi
\ifdefined\listtablename
  \renewcommand*\listtablename{List of Tables}
\else
  \newcommand\listtablename{List of Tables}
\fi
\ifdefined\figurename
  \renewcommand*\figurename{Figure}
\else
  \newcommand\figurename{Figure}
\fi
\ifdefined\tablename
  \renewcommand*\tablename{Table}
\else
  \newcommand\tablename{Table}
\fi
}
\@ifpackageloaded{float}{}{\usepackage{float}}
\floatstyle{ruled}
\@ifundefined{c@chapter}{\newfloat{codelisting}{h}{lop}}{\newfloat{codelisting}{h}{lop}[chapter]}
\floatname{codelisting}{Listing}
\newcommand*\listoflistings{\listof{codelisting}{List of Listings}}
\makeatother
\makeatletter
\makeatother
\makeatletter
\@ifpackageloaded{caption}{}{\usepackage{caption}}
\@ifpackageloaded{subcaption}{}{\usepackage{subcaption}}
\makeatother
\usepackage{bookmark}
\IfFileExists{xurl.sty}{\usepackage{xurl}}{} % add URL line breaks if available
\urlstyle{same}
\hypersetup{
  pdftitle={How does partisan type influence affective polarization?},
  pdfauthor={Tristan Muno; Thomas König},
  colorlinks=true,
  linkcolor={blue},
  filecolor={Maroon},
  citecolor={Blue},
  urlcolor={Blue},
  pdfcreator={LaTeX via pandoc}}


\title{How does partisan type influence affective
polarization?\thanks{Space for acknowledgements. Wordcount: .}}
\usepackage{etoolbox}
\makeatletter
\providecommand{\subtitle}[1]{% add subtitle to \maketitle
  \apptocmd{\@title}{\par {\large #1 \par}}{}{}
}
\makeatother
\subtitle{A comparative study of 25 European democracies}
\author{Tristan Muno \and Thomas König}
\date{January 27, 2026}
\begin{document}
\maketitle
\begin{abstract}
Space for an abstract.
\end{abstract}


\setstretch{2}
We term identifiers, non-identifiers, and non-partisans.

\section{Roadmap}\label{roadmap}

This study examines variation in affective polarization (AP) along
\textbf{three analytically distinct dimensions}:

\begin{enumerate}
\def\labelenumi{\arabic{enumi}.}
\item
  \textbf{Operationalization of partisanship}\\
  Ingroup defined via \textbf{explicit partisan attachment}
  (identity-based) versus \textbf{vote intention/choice} (behavioral
  anchoring)
\item
  \textbf{Measurement of polarization}\\
  \textbf{Attitudinal polarization} (thermometer-based measures) versus
  \textbf{behavioral polarization} (conjoint outcomes)
\item
  \textbf{External validity and scope conditions}\\
  Cross-national variation in the prevalence and distribution of
  partisan types across European democracies
\end{enumerate}

\begin{center}\rule{0.5\linewidth}{0.5pt}\end{center}

\section{Empirical strategy}\label{empirical-strategy}

\subsection{1. Aggregate (European-level)
patterns}\label{aggregate-european-level-patterns}

First, we establish \textbf{European-level benchmarks} of affective
polarization, conditioning on \textbf{partisan status} and
\textbf{measurement strategy}.

\begin{itemize}
\tightlist
\item
  Outcome: Affective polarization\\
\item
  X-axis: Magnitude of AP\\
\item
  Y-axis: Measurement type (API, Wagner MD, CJ-based AP)\\
\item
  Shape / color: Partisan type (explicit partisan, implicit partisan,
  nonpartisan)
\end{itemize}

\textbf{Measurement logic}:

\begin{itemize}
\tightlist
\item
  \textbf{API and CJ-based AP} rely on \emph{identity-based
  ingroup--outgroup definitions} and allow direct comparison between
  \textbf{attitudinal} and \textbf{behavioral} polarization.
\item
  \textbf{Wagner's measures} capture \emph{perceived system-level
  affective differentiation} and uniquely allow inclusion of
  \textbf{nonpartisans}, providing a non-partisan baseline against which
  partisan polarization can be evaluated.
\end{itemize}

This step clarifies whether differences across partisan types reflect: -
ingroup bias, - behavioral discrimination, - or broader differences in
perceived affective structure.

\subsection{2. Comparative analysis: cross-national
variation}\label{comparative-analysis-cross-national-variation}

Second, we examine how these patterns \textbf{vary across countries}.

\begin{itemize}
\tightlist
\item
  Compare the magnitude and dispersion of AP by partisan type within
  countries\\
\item
  Assess whether identity-based and system-level polarization align or
  diverge cross-nationally\\
\item
  Evaluate how the \textbf{distribution of partisan types} conditions
  observed levels of affective polarization
\end{itemize}

This step establishes the \textbf{external validity} of the
individual-level findings and identifies contextual heterogeneity across
European party systems.

\begin{figure}[H]

\centering{

\pandocbounded{\includegraphics[keepaspectratio]{index_files/figure-latex/code-03_explanal-3.1_descr_anal-fig-impl-expl-distribution-output-1.png}}

}

\caption{\label{fig-impl-expl-distribution}Distribution of partisan
types, by country. Stacked horizontal bars show the within-country share
(\%) of three partisan types: explicit partisans (respondents who
reported a subjective attachement to a party, \(T_i=1\)), implicit
partisans (respondents who reported no attachement but did report a vote
preference or intention, \(T_i=0\)), and respondents who reported
neither (none, \(T_i = \emptyset\)). Percentages sum to 100\% within
each country, with country samples containing about \(1,100\)
respondents each (detailed numbers are reported in appendix section X).}

\end{figure}%

\textsubscript{Source:
\href{https://LS-Konig.github.io/CSAP/code/03_explanal/3.1_descr_anal-preview.html\#cell-fig-impl-expl-distribution}{Code
Notebook 3.1}}

\section{Data}\label{data}

We use data from Hahm, Hilpert, and König
(\citeproc{ref-hahm2024divided}{2024}).

\section{Measurement}\label{measurement}

We compare several measurement strategies frequently employed in the
context of multi-party systems.

For our behavioral experiment, we conceptually follow Reiljan's API

\section{Results}\label{results}

\begin{figure}[H]

\centering{

\pandocbounded{\includegraphics[keepaspectratio]{index_files/figure-latex/code-03_explanal-3.2_ap_measures-fig-eu25-ap-comparison-output-1.png}}

}

\caption{\label{fig-eu25-ap-comparison}Comparison of Behavioral and
Attitudinal Affective Polarization in the EU25. This figure displays
aggregated affective polarization (AP) scores across four different
operationalizations, categorized by partisan attachment type. Behavioral
API (Token) measures the difference in expected token allocations
between co-partisans and out-partisans in a conjoint game. Attitudinal
API represents Reiljan (\citeproc{ref-reiljan2020fear}{2020})'s index,
calculated as the distance between in-group and average out-group
thermometer scores. Attitudinal Distance and Spread follow Wagner
(\citeproc{ref-wagner2021affective}{2021}) metrics, capturing the root
mean square distance from the most-liked party and the overall standard
deviation of party evaluations, respectively. All estimates are weighted
by party vote shares within each country to reflect national party
system compositions and then averaged across 25 European countries. To
ensure comparability, attitudinal thermometer scores (originally 0--100)
have been rescaled to a 0--10 range to match the behavioral token scale.
Explicit partisans are those reporting a psychological attachment to a
party; Implicit partisans are those denying attachment but reporting a
vote choice; Total represents all individuals with a partisan anchor;
and Non-Partisans represent respondents without a reported party anchor.
Error bars indicate 95\% confidence intervals based on cross-national
variance.}

\end{figure}%

\textsubscript{Source:
\href{https://LS-Konig.github.io/CSAP/code/03_explanal/3.2_ap_measures.qmd.html\#cell-fig-eu25-ap-comparison}{Code
Notebook 3.2}}

\section{Limitations}\label{limitations}

Our findings remain limited due to the cross-national nature of our
data.

When weighting affective polarization by party size to estimate country
--- or Europe-level scores, this procedure implicitly assumes that the
probability of any individual encountering and correctly identifying
another partisan is proportional to the size of parties' electorates. In
practice, structural factors may violate this assumption: for example,
partisans of smaller or more radical challenger parties may be more
salient or more easily recognized, such that their social visibility
exceeds their electoral share. Consequently, weighted estimates may
over- or understate true horizontal affective polarization in contexts
where partisan recognition is systematically biased.

\section{References}\label{references}

\phantomsection\label{refs}
\begin{CSLReferences}{1}{0}
\bibitem[\citeproctext]{ref-hahm2024divided}
Hahm, Hyeonho, David Hilpert, and Thomas König. 2024. {``Divided We
Unite: The Nature of Partyism and the Role of Coalition Partnership in
Europe.''} \emph{American Political Science Review} 118 (1): 69--87.
\url{https://doi.org/10.1017/S0003055423000266}.

\bibitem[\citeproctext]{ref-reiljan2020fear}
Reiljan, Andres. 2020. {``{`Fear and Loathing Across Party Lines'}(also)
in Europe: Affective Polarisation in European Party Systems.''}
\emph{European Journal of Political Research} 59 (2): 376--96.
\url{https://doi.org/10.1111/1475-6765.12351}.

\bibitem[\citeproctext]{ref-wagner2021affective}
Wagner, Markus. 2021. {``Affective Polarization in Multiparty
Systems.''} \emph{Electoral Studies} 69: 102199.
\url{https://doi.org/10.1016/j.electstud.2020.102199}.

\end{CSLReferences}

\section{Appendix}\label{appendix}

\subsection{Sample descriptives}\label{sample-descriptives}

\begin{longtable}[]{@{}lrr@{}}

\caption{\label{tbl-nresp-country}Sample composition by country. Numbers
denote respondents.}

\tabularnewline

\toprule\noalign{}
Country & N & Percent \\
\midrule\noalign{}
\endhead
\bottomrule\noalign{}
\endlastfoot
Austria & 1277 & 0.04 \\
Belgium & 1305 & 0.04 \\
Bulgaria & 982 & 0.03 \\
Croatia & 1240 & 0.04 \\
Czech Republic & 1135 & 0.04 \\
Denmark & 1200 & 0.04 \\
Estonia & 944 & 0.03 \\
Finland & 1160 & 0.04 \\
France & 1156 & 0.04 \\
Germany & 1188 & 0.04 \\
Greece & 1161 & 0.04 \\
Hungary & 986 & 0.03 \\
Ireland & 1061 & 0.04 \\
Italy & 1172 & 0.04 \\
Latvia & 1148 & 0.04 \\
Lithuania & 1265 & 0.04 \\
Netherlands & 1221 & 0.04 \\
Poland & 1198 & 0.04 \\
Portugal & 1187 & 0.04 \\
Romania & 1480 & 0.05 \\
Slovakia & 1297 & 0.04 \\
Slovenia & 1135 & 0.04 \\
Spain & 1396 & 0.05 \\
Sweden & 1254 & 0.04 \\
United Kingdom & 1279 & 0.04 \\
Total & 29827 & 0.99 \\

\end{longtable}

\textsubscript{Source:
\href{https://LS-Konig.github.io/CSAP/code/03_explanal/3.1_descr_anal-preview.html\#cell-tbl-nresp-country}{Code
Notebook 3.1}}

\begin{longtable}[]{@{}lrrr@{}}

\caption{\label{tbl-nresp-country-gender}Sample composition by country
and gender. Numbers denote respondents.}

\tabularnewline

\toprule\noalign{}
Country & Male & Female & Other \\
\midrule\noalign{}
\endhead
\bottomrule\noalign{}
\endlastfoot
Austria & 645 & 630 & 2 \\
Belgium & 729 & 574 & 2 \\
Bulgaria & 463 & 518 & 1 \\
Croatia & 545 & 694 & 1 \\
Czech Republic & 515 & 618 & 2 \\
Denmark & 690 & 508 & 2 \\
Estonia & 331 & 611 & 2 \\
Finland & 586 & 567 & 7 \\
France & 562 & 594 & 0 \\
Germany & 587 & 597 & 4 \\
Greece & 588 & 572 & 1 \\
Hungary & 493 & 492 & 1 \\
Ireland & 481 & 577 & 3 \\
Italy & 603 & 569 & 0 \\
Latvia & 415 & 733 & 0 \\
Lithuania & 462 & 803 & 0 \\
Netherlands & 642 & 577 & 2 \\
Poland & 540 & 658 & 0 \\
Portugal & 593 & 593 & 1 \\
Romania & 837 & 641 & 2 \\
Slovakia & 550 & 746 & 1 \\
Slovenia & 569 & 566 & 0 \\
Spain & 677 & 718 & 1 \\
Sweden & 648 & 602 & 4 \\
United Kingdom & 635 & 641 & 3 \\
Total & 14386 & 15399 & 42 \\

\end{longtable}

\textsubscript{Source:
\href{https://LS-Konig.github.io/CSAP/code/03_explanal/3.1_descr_anal-preview.html\#cell-tbl-nresp-country-gender}{Code
Notebook 3.1}}

\begin{longtable}[]{@{}
  >{\raggedright\arraybackslash}p{(\linewidth - 14\tabcolsep) * \real{0.1974}}
  >{\raggedright\arraybackslash}p{(\linewidth - 14\tabcolsep) * \real{0.1184}}
  >{\raggedright\arraybackslash}p{(\linewidth - 14\tabcolsep) * \real{0.1184}}
  >{\raggedright\arraybackslash}p{(\linewidth - 14\tabcolsep) * \real{0.1184}}
  >{\raggedright\arraybackslash}p{(\linewidth - 14\tabcolsep) * \real{0.1184}}
  >{\raggedright\arraybackslash}p{(\linewidth - 14\tabcolsep) * \real{0.1184}}
  >{\raggedright\arraybackslash}p{(\linewidth - 14\tabcolsep) * \real{0.1184}}
  >{\raggedleft\arraybackslash}p{(\linewidth - 14\tabcolsep) * \real{0.0658}}@{}}

\caption{\label{tbl-nresp-country-agegroup}Sample composition by country
and age group. Numbers denote respondents.}

\tabularnewline

\toprule\noalign{}
\begin{minipage}[b]{\linewidth}\raggedright
Country
\end{minipage} & \begin{minipage}[b]{\linewidth}\raggedright
18 to 25
\end{minipage} & \begin{minipage}[b]{\linewidth}\raggedright
26 to 35
\end{minipage} & \begin{minipage}[b]{\linewidth}\raggedright
36 to 45
\end{minipage} & \begin{minipage}[b]{\linewidth}\raggedright
46 to 55
\end{minipage} & \begin{minipage}[b]{\linewidth}\raggedright
56 to 65
\end{minipage} & \begin{minipage}[b]{\linewidth}\raggedright
66 to 75
\end{minipage} & \begin{minipage}[b]{\linewidth}\raggedleft
\textgreater{} 75
\end{minipage} \\
\midrule\noalign{}
\endhead
\bottomrule\noalign{}
\endlastfoot
Austria & 144 & 177 & 249 & 263 & 286 & 142 & 0 \\
Belgium & 197 & 132 & 160 & 235 & 333 & 239 & 3 \\
Bulgaria & 57 & 195 & 231 & 250 & 212 & 33 & 0 \\
Croatia & 119 & 251 & 272 & 331 & 205 & 47 & 0 \\
Czech Republic & 88 & 187 & 227 & 219 & 281 & 126 & 0 \\
Denmark & 132 & 134 & 139 & 208 & 321 & 249 & 4 \\
Estonia & 56 & 168 & 164 & 283 & 265 & 7 & 0 \\
Finland & 125 & 178 & 210 & 242 & 259 & 137 & 1 \\
France & 123 & 176 & 242 & 267 & 279 & 51 & 0 \\
Germany & 125 & 184 & 200 & 240 & 313 & 117 & 0 \\
Greece & 77 & 202 & 376 & 318 & 136 & 34 & 0 \\
Hungary & 51 & 175 & 196 & 166 & 273 & 117 & 1 \\
Ireland & 128 & 216 & 223 & 187 & 169 & 120 & 0 \\
Italy & 104 & 196 & 264 & 195 & 294 & 98 & 2 \\
Latvia & 78 & 251 & 231 & 338 & 239 & 6 & 0 \\
Lithuania & 223 & 311 & 257 & 250 & 212 & 2 & 0 \\
Netherlands & 149 & 126 & 161 & 234 & 340 & 194 & 3 \\
Poland & 200 & 342 & 222 & 187 & 196 & 42 & 0 \\
Portugal & 112 & 260 & 289 & 231 & 218 & 67 & 1 \\
Romania & 128 & 339 & 373 & 347 & 202 & 65 & 0 \\
Slovakia & 139 & 223 & 282 & 290 & 250 & 108 & 1 \\
Slovenia & 97 & 186 & 246 & 284 & 234 & 72 & 0 \\
Spain & 125 & 293 & 355 & 305 & 228 & 66 & 1 \\
Sweden & 124 & 165 & 156 & 234 & 297 & 261 & 3 \\
United Kingdom & 121 & 184 & 196 & 234 & 268 & 250 & 1 \\
Total & 3022 & 5251 & 5921 & 6338 & 6310 & 2650 & 21 \\

\end{longtable}

\textsubscript{Source:
\href{https://LS-Konig.github.io/CSAP/code/03_explanal/3.1_descr_anal-preview.html\#cell-tbl-nresp-country-agegroup}{Code
Notebook 3.1}}

\subsection{Experimental setup}\label{experimental-setup}

Before the behavioral games, Hahm, Hilpert, and König
(\citeproc{ref-hahm2024divided}{2024}) presented respondents a short
background information overview and instructions. For the dictator game,
these were: \emph{This game is played by pairs of individuals.}
\emph{Each pair is made up of a Player 1 and a Player 2.} \emph{Each
player will have some information about the other player, but you will
not be told who the other players are during or after the experiment.}
\emph{The game is conducted as follows: A sum of 10 tokens will be
provisionally allocated to Player 1 at the start of each round.}
\emph{Player 1 will then decide how much of the 10 tokens to offer to
Player 2.} \emph{Player 1 could give some, all, or none of the 10
tokens.} \emph{Player 1 keeps all tokens not given to Player 2.}
\emph{Player 2 gets to keep all the tokens Player 1 offers.} \emph{You
will play this game three times with three different people.} In the
trust game, the provided information and instruction were: \emph{This
game is played by pairs of individuals.} \emph{Each pair is made up of a
Player 1 and a Player 2.} \emph{Each player will have some information
about the other player, but you will not be told who the other players
are during or after the experiment.} \emph{Each player will receive 10
tokens.} \emph{Player 1 then has the opportunity to give a portion of
his or her 10 tokens to Player 2.} \emph{Player 1 could give some, all,
or none of the 10 tokens.} \emph{Whatever amount Player 1 decides to
give to Player 2 will be tripled before it is passed on to Player 2.}
\emph{Player 2 then has the option of returning any portion of this
tripled amount to Player 1.} \emph{Then, the game is over.} \emph{Player
1 receives whatever he or she keeps from the original 10 tokens, plus
anything returned to him or her by Player 2. Player 2 receives their
original 10 tokens, plus whatever he or she keeps after returning any
portion of the tripled amount to Player 1.} \emph{You will play this
game three times, with three different people.} \emph{The more tokens
you obtain, the more successful you will be.}

In both games respondents were shown a tabular overview of Player 2
after the instructions. Figure~\ref{fig-example-profile} shows an
example of such a profile along with the interface respondents were
provided to assign the 10 tokens. Each round, a new profile was
displayed to respondents.

\begin{figure}

\centering{

\pandocbounded{\includegraphics[keepaspectratio]{images/cj-example-profile.png}}

}

\caption{\label{fig-example-profile}Example of potential co-player
profile.}

\end{figure}%

\subsection{Distribution of Y}\label{distribution-of-y}

\begin{figure}[H]

\centering{

\pandocbounded{\includegraphics[keepaspectratio]{index_files/figure-latex/code-03_explanal-3.1_descr_anal-fig-distr-y-output-2.png}}

}

\caption{\label{fig-distr-y}Distribution of token allocation (Y) by
game. Dictator game: \(Mean = 3.41\), \(median = 4\), \(SD = 2.35\).
Trust game: \(Mean = 3.48\), \(median = 4\), \(SD = 2.49\)}

\end{figure}%

\textsubscript{Source:
\href{https://LS-Konig.github.io/CSAP/code/03_explanal/3.1_descr_anal-preview.html\#cell-fig-distr-y}{Code
Notebook 3.1}}

\subsection{Distribution of T and R}\label{distribution-of-t-and-r}

\begin{longtable}[]{@{}llll@{}}

\caption{\label{tbl-t-r-distr}}

\tabularnewline

\toprule\noalign{}
der\_partisan\_type & None & Co & Out \\
\midrule\noalign{}
\endhead
\bottomrule\noalign{}
\endlastfoot
0 & 5123 & 1948 & 18465 \\
1 & 14898 & 30820 & 29366 \\

\end{longtable}

\textsubscript{Source:
\href{https://LS-Konig.github.io/CSAP/code/03_explanal/3.1_descr_anal-preview.html\#cell-tbl-t-r-distr}{Code
Notebook 3.1}}

\subsection{Distribution of Covariates by
T}\label{distribution-of-covariates-by-t}

\begin{longtable}[]{@{}
  >{\raggedright\arraybackslash}p{(\linewidth - 4\tabcolsep) * \real{0.3306}}
  >{\raggedright\arraybackslash}p{(\linewidth - 4\tabcolsep) * \real{0.3306}}
  >{\raggedright\arraybackslash}p{(\linewidth - 4\tabcolsep) * \real{0.3306}}@{}}

\caption{\label{tbl-covariate-distr}}

\tabularnewline

\toprule\noalign{}
\begin{minipage}[b]{\linewidth}\raggedright
{\textbf{Variable}}
\end{minipage} & \begin{minipage}[b]{\linewidth}\raggedright
{\textbf{0}\\
N = 25,536}{\textsuperscript{1}}\strut
\end{minipage} & \begin{minipage}[b]{\linewidth}\raggedright
{\textbf{1}\\
N = 75,084}{\textsuperscript{1}}\strut
\end{minipage} \\
\midrule\noalign{}
\endhead
\bottomrule\noalign{}
\endlastfoot
q\_lrpos2\_z & -0.08 (-0.45, 0.29) & -0.08 (-0.82, 0.66) \\
q\_eupos2\_z & 0.08 (-0.68, 0.46) & 0.08 (-0.68, 0.84) \\
q\_econ\_nativism\_z & 0.15 (-1.03, 0.74) & 0.15 (-1.03, 0.74) \\
q\_cult\_nativism\_z & 0.07 (-1.03, 0.62) & 0.07 (-1.03, 0.62) \\
q\_satis\_demo\_country\_z & 0.30 (-0.81, 1.40) & 0.30 (-0.81, 0.30) \\
q\_understand\_nat\_pol\_z & 0.12 (-0.65, 0.12) & 0.12 (-0.65, 0.90) \\
q\_understand\_eu\_pol\_z & 0.22 (-0.50, 0.22) & 0.22 (-0.50, 0.94) \\
q\_parties\_harm\_z & 0.25 (-0.38, 0.88) & 0.25 (-0.38, 0.88) \\
q\_officials\_talk\_action\_z & 0.41 (-0.33, 1.15) & 0.41 (-0.33,
1.15) \\
q\_politics\_good\_evil\_z & -0.14 (-0.76, 0.48) & -0.14 (-0.76,
0.48) \\
q\_people\_unaware\_z & 0.40 (-0.81, 1.00) & -0.20 (-0.81, 1.00) \\
q\_leaders\_educated\_z & 0.39 (-0.32, 1.10) & 0.39 (-1.03, 1.10) \\
q\_expert\_decisions\_z & 0.16 (-0.49, 0.80) & 0.16 (-0.49, 0.80) \\
q\_listen\_other\_groups\_z & 0.18 (-0.74, 1.10) & 0.18 (-0.74, 1.10) \\
q\_democracy\_compromise\_z & -0.29 (-0.29, 0.57) & -0.29 (-0.29,
0.57) \\
q\_interest\_pol\_country\_z & 0.06 (-0.60, 0.72) & 0.06 (-0.60,
0.72) \\
q\_interest\_pol\_eu\_z & -0.28 (-0.96, 0.39) & 0.39 (-0.28, 1.06) \\
q\_eval\_finance\_household\_z & 0.01 (-0.99, 1.00) & 0.01 (-0.99,
1.00) \\
q\_eval\_job\_z & 0.11 (-0.80, 1.03) & 0.11 (-0.80, 0.11) \\
q\_eval\_econ\_country\_z & -0.14 (-1.03, 0.76) & -0.14 (-1.03, 0.76) \\
q\_eval\_econ\_eur\_z & 0.07 (-0.94, 1.09) & 0.07 (-0.94, 1.09) \\
q\_risk\_taking\_z & 0.11 (-0.56, 0.78) & 0.11 (-0.56, 0.78) \\
q\_future\_discount\_z & -0.17 (-0.84, 0.50) & -0.17 (-0.84, 0.50) \\
q\_edu\_z & -0.08 (-0.73, 0.57) & -0.08 (-0.73, 0.57) \\
q\_age\_z & -0.11 (-0.92, 0.64) & 0.09 (-0.79, 0.91) \\
q\_religion\_en & \begin{minipage}[t]{\linewidth}\raggedright
\hfill\break
\strut
\end{minipage} & \begin{minipage}[t]{\linewidth}\raggedright
\hfill\break
\strut
\end{minipage} \\
~~~~catholic & 8,606 (34\%) & 27,379 (36\%) \\
~~~~no religion & 9,462 (37\%) & 25,678 (34\%) \\
~~~~protstnt & 1,881 (7.4\%) & 7,570 (10\%) \\
~~~~other religion & 5,470 (21\%) & 13,791 (18\%) \\
~~~~muslim & 114 (0.4\%) & 664 (0.9\%) \\
q\_perc\_class & \begin{minipage}[t]{\linewidth}\raggedright
\hfill\break
\strut
\end{minipage} & \begin{minipage}[t]{\linewidth}\raggedright
\hfill\break
\strut
\end{minipage} \\
~~~~Working class & 5,307 (22\%) & 15,198 (21\%) \\
~~~~Lower middle class & 4,866 (20\%) & 13,924 (19\%) \\
~~~~Middle class & 12,060 (49\%) & 35,019 (48\%) \\
~~~~Upper middle class & 2,055 (8.4\%) & 7,622 (10\%) \\
~~~~Upper class & 191 (0.8\%) & 1,310 (1.8\%) \\
q\_rural\_urban & \begin{minipage}[t]{\linewidth}\raggedright
\hfill\break
\strut
\end{minipage} & \begin{minipage}[t]{\linewidth}\raggedright
\hfill\break
\strut
\end{minipage} \\
~~~~Rural area or village & 5,819 (23\%) & 17,093 (23\%) \\
~~~~Small or middle sized town & 9,069 (36\%) & 28,418 (38\%) \\
~~~~Large town & 10,573 (42\%) & 29,298 (39\%) \\
q\_gender & \begin{minipage}[t]{\linewidth}\raggedright
\hfill\break
\strut
\end{minipage} & \begin{minipage}[t]{\linewidth}\raggedright
\hfill\break
\strut
\end{minipage} \\
~~~~Male & 10,939 (43\%) & 39,456 (53\%) \\
~~~~Female & 14,565 (57\%) & 35,514 (47\%) \\
~~~~Other & 32 (0.1\%) & 114 (0.2\%) \\
{\textsuperscript{1}} {Median (Q1, Q3); n (\%)} & & \\

\end{longtable}

\textsubscript{Source:
\href{https://LS-Konig.github.io/CSAP/code/03_explanal/3.1_descr_anal-preview.html\#cell-tbl-covariate-distr}{Code
Notebook 3.1}}

\subsection{Robustness}\label{robustness}

\section*{Eidesstattliche Erklärung -- Statutory
Declaration}\label{eidesstattliche-erkluxe4rung-statutory-declaration}

\noindent Hiermit versichere ich, dass diese Arbeit von mir persönlich
verfasst ist und dass ich keinerlei fremde Hilfe in Anspruch genommen
habe. Ebenso versichere ich, dass diese Arbeit oder Teile daraus weder
von mir selbst noch von anderen als Leistungsnachweise andernorts
eingereicht wurden. Wörtliche oder sinngemäße Übernahmen aus anderen
Schriften und Veröffentlichungen in gedruckter oder elektronischer Form
sind gekennzeichnet. Sämtliche Sekundärliteratur und sonstige Quellen
sind nachgewiesen und in der Bibliographie aufgeführt. Das Gleiche gilt
für graphische Darstellungen und Bilder sowie für alle Internet-Quellen.
Ich bin ferner damit einverstanden, dass meine Arbeit zum Zwecke eines
Plagiatsabgleichs in elektronischer Form anonymisiert versendet und
gespeichert werden kann. Mir ist bekannt, dass von der Korrektur der
Arbeit abgesehen und die Prüfungsleistung mit „nicht ausreichend``
bewertet werden kann, wenn die Erklärung nicht erteilt wird.

\noindent I hereby declare that the paper presented is my own work and
that I have not called upon the help of a third party. In addition, I
affirm that neither I nor anybody else has submitted this paper or parts
of it to obtain credits elsewhere before. I have clearly marked and
acknowledged all quotations or references that have been taken from the
works of other. All secondary literature and other sources are marked
and listed in the bibliography. The same applies to all charts, diagrams
and illustrations as well as to all Internet sources. Moreover, I
consent to my paper being electronically stores and sent anonymously in
order to be checked for plagiarism. I am aware that the paper cannot be
evaluated and may be graded ``failed'' (``nicht ausreichend'') if the
declaration is not made.

\vspace{2cm}
\noindent
\parbox{5cm}{
  \hrulefill\\
  Place, Date
}
\hfill
\parbox{5cm}{
  \hrulefill\\
  Signature
}




\end{document}
